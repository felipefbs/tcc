\chapter{Considerações Finais}
\label{cap:conclusao}
A analise dos dados, apresentada sa seção \ref{resultados}, indica que a modulação SUN-OQPSK apresenta melhor performance em relação as outras modulações utilizadas. Porém, ainda não apresenta uma performance constante em todos os casos abordados, bem como as modulações SUN-FSK e SUN-OFDM chegam a apresentar valores de PDR próximos a 0\% em algumas situações. Os resultados obtidos motivam a novas abordagens para a implementação como a utilização de diversidade, por exemplo diversidade de receptores ou a utilização de diversidade de modulação que a tecnologia utilizada proporciona.


\section{Sugestões para Trabalhos Futuros}
\label{sec:futuros}
Esse trabalho foi uma extensão direta do experimento realizado em \cite{tuset2020dataset} e se propós a seguir as mesmas diretrizes utilizadas neste experimento e há diversos pontos a serem melhorados e novas abordagens podem ser feitas para o ambiente proposto nesse trabalho. No total, é possível ter 33 configurações diferentes no transceptor de rádio utilizado, neste trabalho foram utilizados apenas 3 configurações, sendo possível então ter um trabalho para analisar como essas diversas configurações de taxa de transmissão, índice de modulação, canais de frequência entre outros são impactados na transmissão sem fio no ambiente.

A principal deixa para trabalhos futuros é a implementação de utilização de pacotes \emph{acknowledgment}, pacotes de reconhecimento. No qual, ao receber um pacote o receptor envia uma mensagem ao transmissor confirmando o recebimento. Isto permitiria ao transmissor analisar quais modulações estão sendo melhor recebidas e determinar por si só quais as melhores modulações para aquele instante de tempo. Possibilitando uma melhoria nos valores de PDR.

Outras sugestões, e que não foram objetivos desse trabalho, são: analisar o consumo energético dos dispositivos e verificar como pode ser otimizado de acordo com as diversas configurações. Este tópico é bastante importante devido ao tipo de problema que o padrão IEEE 802.15.4g SUN tenta resolver que é a utilização destes dispositivos por longos períodos de tempo utilizando como fonte energética baterias; Utilização de múltiplos receptores/\emph{gateways} que resultaria em uma maior recepção de dados, mas trarias problemas como entradas duplicadas no banco de dados; Replicação deste experimento em outros tipos de cenários, por exemplo, cenários rurais que, teoricamente, possuem um ambiente mais favorável para a comunicação via rádio ou cenários urbanos onde efeitos de propagação por múltiplos caminhos são acentuados devido aos diversos prédios e ao carros que circulam pela cidade.
