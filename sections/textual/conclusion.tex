\chapter{Considerações Finais}
\label{cap:conclusao}
A análise dos dados, apresentada no Capítulo \ref{resultados}, indica que a modulação SUN-OQPSK apresenta melhor desempenho em relação às outras modulações utilizadas. Porém, sem desempenho similar em todos os casos abordados, bem como as modulações SUN-FSK e SUN-OFDM chegam a apresentar valores de PDR próximos a 0\% em algumas situações. Os resultados obtidos motivam novas abordagens para a implementação, como a utilização de diversidade de receptores. E como principal contribuição, este trabalho apresenta um conjunto de dados que podem ser utilizados para o avanço dos estudos de diversidade de modulação.

\section{Sugestões para Trabalhos Futuros}
\label{sec:futuros}
Este trabalho é uma extensão daquele realizado em \cite{tuset2020dataset} e segue as mesmas diretrizes utilizadas nele, com diversos pontos a serem melhorados e novas abordagens podem ser feitas para o ambiente estudado neste trabalho. No total, é possível ter 33 configurações diferentes no transceptor de rádio utilizado. Neste trabalho foram utilizadas apenas 3 configurações (uma para cada tipo de modulação), sendo possível analisar como essas diversas configurações de taxa de transmissão, índice de modulação, canais de frequência, entre outras impacta a transmissão sem fio no ambiente estudado.

A principal linha de trabalhos futuros é a implementação de utilização de pacotes \emph{acknowledgment} ou pacotes de reconhecimento. Em que, ao receber um pacote o receptor envia uma mensagem ao transmissor confirmando o recebimento. Isso permite ao transmissor analisar quais modulações estão sendo melhor recebidas e determinar quais as melhores modulações para aquele instante, possibilitando uma melhoria nos valores de PDR.

Outras sugestões, e que não foram objetivos deste trabalho, são: analisar o consumo energético dos dispositivos e verificar como pode ser otimizado de acordo com as diversas configurações. Este tópico é bastante importante devido ao tipo de problema que o padrão IEEE 802.15.4g SUN tenta resolver, que é a utilização destes dispositivos por longos intervalos utilizando como fonte energética baterias; Utilização de múltiplos receptores/\emph{gateways}, que resultaria em melhor recepção de dados, mas com problemas como entradas duplicadas no banco de dados; Replicação deste experimento em outros tipos de cenários, por exemplo, cenários rurais que, teoricamente, são um ambiente mais favorável para a comunicação via rádio ou cenários urbanos onde efeitos de propagação por múltiplos caminhos são acentuados devido aos diversos prédios e ao carros que circulam pela cidade.