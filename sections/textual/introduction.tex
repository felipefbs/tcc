\chapter*[Introdução]{Introdução}
\addcontentsline{toc}{chapter}{Introdução}
% ---

IoT é um conceito que vem ganhado força ao longo dos anos, e isto só foi possível após anos de estudo, prototipação e desenvolvimento de várias tecnologias, padrões e protocolos. Sendo os principais pontos, a miniaturização de processadores e sensores; melhoramento de baterias e otimização do uso destas; definição de novos protocolos de rede; e aumento da robustez de protocolos de comunicação sem-fio.

A Internet das Coisas, mais conhecida pelo seu acrônimo em inglês IoT, foi cunhada pelo engenheiro britânico Kevin Ashton no final dos anos 1990 onde ele, trabalhando para a P\&G, pensou na possibilidade de que os produtos da empresa estivessem munidos de identificadores e capazes de estabelecer comunicação sem fio e se comunicando através da internet, que na época estava se estabelecendo, criando assim uma internet onde as coisas estivesse conectadas\cite{KA_IOT}. Fazendo então que os computadores fossem capazes de rastrear e identificar tudo, reduzindo desperdícios e custos e identificando no momento certo quando substituir ou reparar um produto\cite{lopezIOT}.

Atualmente há diversas aplicações de IoT, desde grandes implementações como cidades inteligentes como em \cite{sotres2017practical}, onde na cidade de Santander na Espanha é implantado por toda a cidade sensores para analisar a qualidade do ar dos habitantes. Até aplicações de saúde como em \cite{zhang2015remote}, onde é coletado e analisado em tempo real informações de pressão sanguínea e peso corporal do paciente então é aplicado conceitos de aprendizado de maquina para verificar a probabilidade do paciente ter um evento de insuficiência cardíaca.

Após o termo ser criado em 1999, foi necessário anos de evolução tecnológica para a atual popularidade do conceito. Por exemplo, a criação da plataforma de desenvolvimento de hardware aberta Arduino em 2005\cite{OC_ARDUINO}, tornou fácil o estudo e a prototipação de itens de baixo custo. \ff{adicionar exemplos como o IPv6, baterias e os protocolos de Sem Fio}

A implementação de uma aplicação IoT necessita de uma rede de nó sensores distribuídas geralmente conectadas a um ou mais nó central, também conhecido como gateway, que tem por finalidade encaminhar os dados coletados para processamento. Para tal implementação, existem duas abordagens clássicas, conexões cabeadas entre os nós sensores e a utilização de redes sem fio\cite{gomes2017estimaccao}. Em vantagem a redes sem fio, as conexões cabeadas apresentam maior confiabilidade nas camada física. Porém, a utilização de redes sem fio se destaca, em relação a redes cabeadas, nos quesitos de flexibilidade, custo de implantação , facilidade e rápida de implementação e de manutenção\cite{gungor2009industrial}.

As vantagens que estas Redes de Sensores Sem Fio, também conhecidas como RSSF, apresentam então fazem elas se destacar, porém apresentam ainda desafios para uma implementação robusta. Pois o meio onde ocorre a comunicação via rádio é intrinsecamente não confiável, pode apresenta interferências, possui ruídos e alguns outros problemas clássicos de comunicação sem fio como sombreamento e propagação por multi percurso. Além de apresentar baixa taxa de transferência e valores de latência alto\cite{gomes2017estimaccao}.

Para contornar os diversos problemas da comunicação sem fio padrões e tecnologias foram desenvolvidos. Em cenários industriais, são proeminentes o WirelessHart, tecnologia RSSFs baseada no padrão Hart\cite{WIHART} e o padrão ISA100.11a. Ambos padrões são baseados na camada física do padrão IEEE802.15.4, porém apresentam diferentes implementações de Controle de Acesso ao Meio\cite{gomes2017estimaccao}, MAC. Além do cenário industrial, o Zigbee é uma tecnologia de baixo consumo energético, baixas taxas de transferência e de baixo custo baseado no protocolo de redes sem fio IEEE802.15.4 que tem por objetivo aplicações de controle remoto e automação\cite{ergen2004zigbee}. 6LoWPAN é um protocolo definido na RFC6282 pela IETF que pretende distribuir endereços de IP versão 6 para dispositivos de RSSF, sendo o principal alvo dispositivos que tem tecnologias baseadas no IEEE802.15.4, facilitando assim a conexão destas redes à internet já que podem compartilhar os mesmos endereços IPv6\cite{olsson20146lowpan}. LoRa é uma tecnologia que utiliza uma técnica de modulação proprietária chamada Chirp Spread Spectrum, que permite uma melhor sensibilidade e uma largura de banda fixa em troca de uma menor taxa de transferência\cite{sinha2017survey}. NarrowBand-IoT é uma tecnologia integrada com o padrão de comunicações moveis LTE, porém simplificada para reduzir custos e complexidade dos dispositivos, minimizando consumo de bateria\cite{sinha2017survey}.

Ao padrão IEEE802.15.4, desde sua criação em 2003, vem sendo anexado emendas criando assim novas possibilidades de usos e tecnologias. Com a emenda de 2015 ao padrão foi definida três modulações para utilização na camada física e mudanças para a MAC\cite{chang2012ieee}. Como forma. Em \cite{tuset2020reliability} foi proposto o uso destas três modulações em conjunto para melhorar a qualidade das comunicações, a partir de uma maior taxa de entregas de pacotes. Este artigo foi realizado em um cenário industrial onde os desafios para a comunicação sem fio estão, principalmente, em interferências de fontes externas e propagação por multi caminho. Pensando em analisar os efeitos da propagação de ondas de rádio em outro cenário, este trabalho então se propõe a analisar a comunicação sem fio de dispositivos em um ambiente predial, que possui como maior fonte de problemas para a comunicação a falta de linha de visada. Estudando como cada modulação se comporta a partir da verificação da taxa de entrega de pacotes, PDR, em cada modulação.