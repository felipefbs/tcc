\chapter[Introdução]{Introdução}
\label{cap:intro}

IoT é um conceito que vem ganhado força ao longo dos anos, graças a estudo, prototipação e desenvolvimento de diversas protocolos e tecnologias. Os pontos chave para o seu desenvolvimento foram a miniaturização de processadores e sensores; melhoramento de baterias e otimização do uso destas; definição de novos protocolos de rede; e aumento da robustez de protocolos de comunicação sem-fio.

A Internet das Coisas, mais conhecida pelo seu acrônimo em inglês IoT(Internet of Things), foi cunhada pelo engenheiro britânico Kevin Ashton no final dos anos 1990 onde ele, trabalhando para a P\&G, pensou na possibilidade de que os produtos da empresa estivessem munidos de identificadores e capazes de estabelecer comunicação através da internet, que na época estava se estabelecendo, criando assim uma internet onde as coisas estivesse conectadas\cite{KA_IOT}. Assim, os computadores se tornariam capazes de rastrear e identificar tudo, podendo reduzir desperdícios, minimizar custos e identificar o momento certo quando substituir ou reparar um produto\cite{lopezIOT}.

Atualmente há diversas aplicações de IoT, desde implementações em larga escala como cidades inteligentes conforme descrito em \cite{sotres2017practical}, onde em Santander na Espanha é implantado por toda a cidade nós com múltiplos sensores afim de disponibilizar uma plataforma de testes. Aplicações de smart campus como apresentada em \cite{wang2017performance} que mostra a aplicação de uma rede de sensores sem fio para verificar a qualidade do ar. Aplicações de saúde como em \cite{zhang2015remote}, onde é coletado e analisado em tempo real informações de pressão sanguínea e peso corporal do paciente, estes dados são utilizados então para verificar a probabilidade do paciente ter um evento de insuficiência cardíaca, utilizando técnicas de aprendizado de maquina.

Após o termo ser cunhado em 1999, foram necessários anos de evolução tecnológica para a atual popularidade do conceito. Por exemplo, a criação da plataforma de desenvolvimento de hardware aberta Arduino em 2005\cite{OC_ARDUINO}, tornou fácil o estudo e a prototipação de itens de baixo custo. \ff{adicionar exemplos como o IPv6, baterias e os protocolos de Sem Fio}

A implementação de uma aplicação IoT necessita de uma rede de nó sensores distribuídas geralmente conectadas a um nó central, conhecido como nó gateway, que tem por finalidade encaminhar os dados coletados para processamento. Para tal implementação existem duas abordagens clássicas, conexões cabeadas entre os nós sensores e a utilização de redes sem fio\cite{gomes2017estimaccao}. Como vantagem em relação à redes sem fio, as conexões cabeadas apresentam maior confiabilidade na camada física. Em contrapartida, a utilização de redes sem fio se destaca, em relação a redes cabeadas, nos quesitos de flexibilidade, custo de implantação, facilidade e rápidez na implementação e na manutenção\cite{gungor2009industrial}.

As vantagens que Redes de Sensores Sem Fio, RSSF, fazem estas se destacarem na implementação de sistemas IoT, porém há ainda desafios para uma implementação robusta de comunicações sem fio, pois o meio de transmissão é caótico e pouco confiável. Apresenta interferências, é ruidoso e possui altos valores de latência, podendo também ter problemas como sombreamento e propagação por multi percurso\cite{gomes2017estimaccao}.

Padrões e tecnologias foram desenvolvidos com a intenção de minimizar os problemas intrínsecos da comunicação sem fio, a seguir estão algumas das principais tecnologias aplicadas em IoT. Em cenários industriais, são proeminentes o WirelessHart, tecnologia RSSFs baseada no padrão Hart\cite{WIHART} e o padrão ISA100.11a. Ambos padrões são baseados na camada física do padrão IEEE802.15.4, porém apresentam diferentes implementações de Controle de Acesso ao Meio\cite{gomes2017estimaccao}, MAC. Além do cenário industrial, o Zigbee é uma tecnologia de baixo consumo energético, baixas taxas de transferência e de baixo custo baseado no protocolo de redes sem fio IEEE802.15.4 que tem por objetivo aplicações de controle remoto e automação\cite{ergen2004zigbee}. 6LoWPAN é um protocolo definido na RFC6282 pela IETF que pretende distribuir endereços da versão 6 do IP para dispositivos de RSSF, sendo o principal alvo dispositivos que tem tecnologias baseadas no IEEE802.15.4(Zigbee), facilitando assim a conexão destas redes à internet já que podem compartilhar os mesmos endereços IPv6\cite{olsson20146lowpan}. Ao padrão IEEE802.15.4, desde sua consolidação em 2003, vem sendo anexado emendas criando assim novas possibilidades de usos e tecnologias. Com a emenda ``g'' de 2015 ao padrão foi definida três modulações para utilização na camada física e mudanças para a MAC\cite{chang2012ieee}. LoRa é uma tecnologia que utiliza uma técnica de modulação proprietária chamada Chirp Spread Spectrum, que permite uma melhor sensibilidade e uma largura de banda fixa em troca de uma menor taxa de transferência\cite{sinha2017survey}. NarrowBand-IoT é uma tecnologia integrada com o padrão de comunicações moveis LTE, porém simplificada para reduzir custos e complexidade dos dispositivos, minimizando consumo de bateria\cite{sinha2017survey}.


\section{Justificativa e Relevância do Trabalho}
\label{sec:justificativa}
Em \cite{tuset2020dataset} foi proposto o uso das três modulações da emenda ``g'' do padrão IEEE 802.15.4 em conjunto para melhorar a qualidade das comunicações, baseado na taxa de entregas de pacotes. Este artigo foi realizado em um cenário industrial onde os desafios para a comunicação sem fio estão, principalmente, em interferências de fontes externas e propagação por multi caminho. Pensando em analisar os efeitos da propagação de ondas de rádio em outro cenário, este trabalho então se propõe a analisar a comunicação sem fio de dispositivos que implementam as modulações do padrão IEEE 802.15.4g em um ambiente predial, também denominado como Smart Building, que possui como maior fonte de problemas para a comunicação a falta de linha de visada. Estudando como cada modulação se comporta a partir da verificação da taxa de entrega de pacotes, PDR, em cada modulação. E portanto, analisando a viabilidade de implementação de uma RSSF a partir da tecnologia utilizada.

\section{Objetivos}
\label{sec:objetivos}

\subsection{Objetivo Geral}
\label{subsec:objGeral}
Ensaiar uma aplicação de IoT afim de coletar as informações das transmissões realizadas através das modulações definidas pelo padrão IEEE 802.15.4g para analisar o comportamento destas no cenário de Smart Building.

% Coletar e analisar os dados experimentais de transmissões realizadas entre os dispositivos openmote que implementam as modulações definidas no padrão IEEE 802.15.4g SUN espalhados pelo prédio dos professores no campus Campina Grande.

\subsection{Objetivos Específicos}
\label{subsec:objespecificos}
\begin{itemize}
    \item Implementar uma rede sem fio utilizando transceptores das modulações do IEEE 802.15.4g
    \item Coletar, armazenar e disponibilizar os dados relativas as transmissões realizadas
    \item Analisar os dados e gerar informações a respeito da comunicação entre dispositivos neste cenário
\end{itemize}


\section{Metodologia}
\label{sec:metodologia}
Para alcançar os objetivos acima, foram realizados os seguintes passos:

% Estudei a plataforma de desenvolvimento do Openmote
% Fiz as mudanças no código de Pere
% fiz o código python dos gateways/persistência dos dados no influx
% Colocar pra rodar o experimento
% analise dos dados

\begin{itemize}
    \item \refact{Estudo e implementação dos dispositivos: Os dispositivos utilizados neste projeto possuem o código disponível em um repositório github \cite{pere2019}. Bem como algumas alterações no código fonte para esta aplicação.}
    \item \refact{Persistência dos dados: Desenvolvimento de um script responsável por coletar e persistir os dados gerados em um banco de dados InfluxDB}
    \item \refact{Rodar o experimento: Distribuição dos nós transmissores pelo prédio e instalação dos softwares e scripts necessarios para persistencia dos dados.}
    \item \refact{Analise dos resultados: A partir dos dados coletados e salvos localmente, foram feitas analises e gerado informações as quais estarão na seção de resultados}
\end{itemize}

\section{Organização do Documento}
\label{sec:organizacao}
\ff{A ser feito quando o documento tiver pronto}