\chapter{Fundamentação Teórica}
\label{fundamentacao}
\todo{Texto inicial do capitulo}

\section{Classificação de redes}
\label{classRedes}
Segundo Tanembaum em \cite{tanembaum2011}, redes de telecomunicações são classificadas, comumente, de acordo com sua escala de abrangência. Esta classificação é importante pois diferentes tecnologias e padrões são criados para atacar problemas específicos destas redes. Abaixo serão apresentadas as classificações mais relevantes para o escopo deste trabalho de acordo Rochol em \cite{rochol2018sistemas}.

\todo{imagem das redes PAN, LAN, WAN etc}

\subsection{Classificação de redes sem fio: escala de abrangência}
\subsubsection*{Redes sem fio pessoais - WPAN}
Inicialmente estas redes foram definidas para possuir alcance limitado com a principal função de facilitar a conectividade de dispositivos periféricos ao computador ou telefone celular através de uma conexão sem fio. O padrão IEEE 802.15 especifica a arquitetura destas redes sem fio, também denominadas, piconets. A principal  tecnologia comercial de redes WPANs é o Bluetooth, baseado no padrão IEEE 802.15.1.

O conceito de redes WPAN foi expandido para designar qualquer tipo de rede sem fio que atua em área restrita interconectando dispositivos, periféricos, atuadores ou um conjunto de sensores, geralmente com a capacidade de se auto-organizarem. E a partir dessa expansão é possível aplicações como redes de sensores sem fio, sistemas de anticolisão e de condução automática em veículos, automação industrial, prédios inteligentes, monitoração de pacientes entre várias outras.

Então, basicamente temos redes WPAN para interconexão de dispositivos onde o Bluetooth é a tecnologia mais proeminente no mercado. E Redes de Sensores Sem Fio WPAN implementadas, principalmente, com a tecnologia ZigBee, que implementa a camada física padronizada pela IEEE 802.15.4.


\subsubsection*{Redes sem fio locais - WLAN}
São redes com a finalidade de conectar computadores de casas, escritório ou prédio em uma rede privada. Neste tipo de rede, a tecnologia Wi-Fi se tornou extremamente popular sendo adotado praticamente em todo computador e dispositivos moveis como \emph{smartphones} e \emph{tablets}. O Wi-Fi é baseado no padrão IEEE 802.11.

Inicialmente, foi definido no padrão a utilização da faixa de frequência ISM de 2.4GHz. Com novas emendas do padrão IEEE 802.11 foi adicionado a faixa de frequência ISM 5GHz possibilitando maiores taxas de transferência para dispositivos próximos. Como utiliza uma faixa de frequência livre, é comum ter interferência com outras redes Wi-Fi e com redes formadas por outras tecnologias que atuam na mesma faixa de frequência, como é o caso do Bluetooth e ZigBee que atuam na faixa de frequência de 2.4GHz.

\subsubsection*{Redes celulares de telefonia e dados}
Estas redes compreendem diversos sistemas de telefonia conectados e ganharam bastante popularidade nas ultimas décadas devido, principalmente pela redução de custos e aumento nas taxas de transferência oferecidas pelas operadoras. Exemplos mais recentes de telefonia celular são GSM, EDGE e LTE que fornecem a segunda, terceira e quarta geração(2G, 3G e 4G) de telefonia celular.

Como será abordado na seção \ref{padrõesSF} há aplicações que se utilizam da infraestrutura GSM e LTE para aplicações de RSSF.


\subsubsection*{Redes de loga distância com baixo consumo energético - LPWAN}
Algumas aplicações IoT possuem requisitos muito específicos como atuação por uma grande extensão, dispositivos que possam passar longos períodos utilizando baterias e um bom custo beneficio. Tecnologias como ZigBee e Bluetooth não apresentam ua grande área de cobertura. Aplicações de telefonia celular apesar de conseguirem abranger uma grande área, possuem um consumo energético excessivo. Então, estas aplicações IoT impulsionaram a criação de novos padrões e tecnologia para um novo tipo de rede denominada de redes de longa distância com baixo consumo energético ou Low Power Wide Area Network(LPWAN).

Dispositivos que implementam redes de baixo consumo energético ganharam popularidade na indústria e nas comunidades de pesquisa pois estabelecem comunicação sem fio de longas distâncias, cerca de até alguns quilômetros, com baterias que podem durar anos com uma única carga \cite{mekki2019comparative}.

A utilização deste tipo de rede se tornou fundamental para a implementação de RSSFs, oferecendo confiabilidade e baixo custo de implantação e manutenção para redes sem fio. Na seção \ref{padrõesSF} serão apresentadas alguns dos principais padrões e tecnologias utilizadas nestas redes.

\section{Modelo de camadas OSI}
\label{osi}
\subsection{Camada física}
\subsection{Camada de controle de acesso ao meio}

\section{Problemas enfrentados pela comunicação via rádio}


\section{Padrões e tecnologias de comunicação para redes LPWAN}
\label{padrõesSF}
Segundo Tanembaum em \cite{tanembaum2011}, sem coordenação e cooperação entre as fabricantes de dispositivos haveria o caos completo onde não seria possível a interoperabilidade de sistemas. Sistemas IoT, como demonstrado em \cite{sotres2017practical}, podem ter apresentar sistemas diferentes com vários dispositivos. Portanto a padronização das telecomunicações são imprescindíveis atualmente. Permitindo dispositivos de diferentes fabricantes consigam se comunicar, não importando quem produziu a placa de rede, os cabos, roteadores ou comutadores.

\subsection{LoRa}
LoRa, abreviação para Long Range, é uma tecnologia que utiliza o padrão LoRaWAN e um protocolo de acesso ao meio de mesmo nome. Modula sinais sub-GHz na faixa ISM utilizando uma técnica de modulação  privada chamada Chirp Spread Spectrum. que espalha um sinal de banda curta por todo um canal de banda larga, esta técnica possibilita uma maior resistência a interferências e torna mais difícil detectar ou obstruir o canal de comunicação. A taxa de transferência pode variar de 300 bits/s a até 50 kbits/s, dependendo do fator de espalhamento utilizado na comunicação. O payload máximo de uma mensagem é 243 bytes \cite{mekki2019comparative}.

A tecnologia LoRa é desenvolvida pela empresa estadunidense SemTech e o padrão LoRaWAN é mantido pelo consorcio de empresas chamado LoRa Alliance.

\subsection{NB-IoT}
NarrowBand IoT é uma tecnologia especificada na versão 13 do 3GPP. Esta tecnologia opera na faixa de frequência celular(LTE e GSM), modulando os sinais utilizando a técnica QPSK. Utiliza faixas de frequência não ocupadas da faixa de frequência LTE. Seu protocolo é baseado em uma versão simplificada do protocolo LTE, que reduz as funcionalidades do LTE para melhor se adequar sua utilização para IoT, por exemplo, funções como monitoramento da qualidade do sinal ou conectividade dupla não são utilizadas, visando diminuir o uso de energia e aumentar a vida útil da bateria. Possui taxa transferência máxima de 200 kbits/s e seu payload máximo é de 1600 bytes.

\subsection{SigFox}
SigFox é uma tecnologia que não possui padronização oficial e é mantida pela empresa homônima. Os nós finais conectados a esta rede enviam para as estações radio-base mensagens utilizando a modulação BPSK utilizando bandas de frequência ultra-estreitas, até 100 Hz, na faixa de frequência ISM. Como utiliza uma faixa de frequência ultra estreita é bastante eficiente energeticamente e bastante resistente à interferência. Porém, não pode oferecer taxas de transferência acima de 100 bits/s.

A empresa SigFox oferece uma solução com conectividade ponta a ponta, o que significa que os dados entre o nó final são transmitidos para as estações rádio-base e vão diretamente para os servidores da empresa. Sendo possível apenas o envio de 140 mensagens por dia e cada mensagem enviada pode ter um payload máximo de 12 bytes.

\subsection{IEEE 802.15.4}
O padrão IEEE 802.15.4 define em seu texto, camadas físicas e de acesso ao meio utilizando a faixa ISM sub-GHz e 2.4 GHz, utilizando, respectivamente, as modulações BPSK e O-QPSK. Possuindo uma taxa de transferência máxima de 250 kbits/s utilizando a faixa de frequência de 2.4 GHz com um tamanho máximo de payload de 127 bytes \cite{munoz2018overview} \cite{gomes2017estimaccao}. Este padrão é utilizado como base para a utilização do ZigBee.

\subsection*{Emenda 'g'}
Em 2015 foi proposto uma revisão que incorpora uma novo esquema de modulações para permitir uma melhor volatilidade entre a faixa de comunicação, ocupação da largura de banda, taxa de transferência de dados e confiabilidade da comunicação para melhor se adequar com a aplicação. O padrão então foi adotado como uma emenda ao padrão de 2003, definindo assim o padrão IEEE 802.15.4g que incluia as modulações SUN(Redes de Utilidades Inteligentes, Smart Utility Network) \cite{tuset2020reliability}.
A seguir uma visão geral das modulações SUN de acordo apresentado em \cite{tuset2020reliability}:
\subsubsection*{SUN-FSK}
Foi incluída no padrão devido a sua eficiência e compatibilidade com sistemas legado. Possui 3 modos de operação para cada uma das faixas de frequência definidas no padrão e possui parâmetros de canal e de modulação, nos quais é possível definir o tipo de modulação, o espaçamento do canal e o índice de modulação. Estes parâmetros definem uma faixa de taxa de transferência de 50 kbits/s a até 50 kbits/s.
\subsubsection*{SUN-OQPSK}
Esta modulação está presente no texto original do padrão e foi estendida nesta emenda para adicionar bandas de frequência e suportar diferentes fatores de espalhamentos para conseguir atingir taxas de transferência entre 6.25 kbits/s a até 500 kbits/s.
\subsubsection*{SUN-OFDM}
Esta modulação consegue prover altas taxas de transferência e maiores faixas de comunicação, enquanto consegue lidar com interferência e propagação por multi-caminho. Utiliza diferentes Esquemas de Modulação e Codificação para alternar em modulações, como BPSK, QPSK e 16-QAM, e esquemas de repetição de frequência para prover uma faixa de taxa de transferência entre 50 kbits/s a até 800 kbits/s em um canal com a largura de banda que varia entre 200kHz e 1.2MHz.

\subsubsection*{Diversidade de Modulação}
Em \cite{gomes2020improving} os autores propõem a utilização das três modulações SUN em um esquema de diversidade de modulação como forma de aumentar a confiabilidade do enlace sem fio.

\section{Parâmetros para avaliação da confiabilidade do enlace sem fio}
\label{paramSF}
Algumas métricas são utilizadas para medir a qualidade de um enlace sem fio, algumas estão mais relacionadas a camada física, como RSSI, e outras estão mais relacionadas a camada de aplicação como PDR. Estas serão melhor detalhadas abaixo.
\subsection*{RSSI}
Received Signal Strength Indicator, Indicador da Força do Sinal Recebido em tradução livre, é uma medida da energia total presente em um sinal de rádio recebido. RSSI é um valor relativo e pode variar de acordo com a fabricante do transceptor de rádio \cite{UNDERSTANDING_RSSI}. Geralmente o valor de RSSI é apresentado em números negativos que vão de -100 até 0, onde valores próximos de -100 são sinais de baixa qualidade e sinais com valores próximos a 0 são de ótima qualidade.

\subsection*{PDR}
Packed Delivery Ratio, Taxa de Entrega de Pacotes, é um indicador de camada de aplicação que relaciona a quantidade de pacotes recebidos pelo receptor pela quantidade de pacotes enviados pelo transmissor. O PDR pode ser um valor importante para medir um enlace de dados, pois dependendo da aplicação há uma taxa minima de entrega de pacotes, por exemplo, algumas aplicações industriais necessitam de um PDR superior a 99.9\%

\subsection*{CCA}

\section{Trabalhos relacionados}
Em \cite{tuset2020dataset} os autores do artigo realizaram um experimento para analisar o comportamento das modulações SUN do padrão IEEE 802.15.4g em um cenário industrial. Em  \cite{munoz2018overview} os autores realizaram um estudo aprofundado sobre a modulação SUN-OFDM no ambiente predial, demonstrando, entre outras coisas, que a modulação SUN-OFDM nas faixas de frequência sub-GHz se é mais vantajosa de utilizar que a modulação OQPSK do padrão IEEE 802.15.4.
