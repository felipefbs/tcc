\chapter{Experimento}
\label{experimento}
Nesta seção será detalhado como foi realizado o experimento realizado. O experimento aconteceu em um dos prédios do campus Campina Grande do Instituto Federal da Paraíba(IFPB). O prédio é constituído de diversas salas utilizadas pelos professores da instituição e de alguns laboratórios, possui 4 andares separados por grossos pisos de concreto armado. Este cenário é particularmente desafiador para um enlace sem fio pois não é possível ter uma linha de visada entre o transmissor e o receptor além de apresentar muito ocorrência de propagação por multi-caminho e pode apresentar fácilmente áreas de sombreamentos de sinais.



De forma geral, o experimento foi realizado utilizando ao total onze dispositivos transmissores enviando mensagens nas modulações SUN para três dispositivos receptores, um para cada modulação em uma topologia estrela. Os nós transmissores, denominados neste texto como Tx, foram espalhados nos quatro andares prédio, posicionados em canaletas e no interior de salas. Os nós receptores, denominados neste texto como Rx, foram posicionados no laboratório do GComPI presente no primeiro piso do prédio.

Abaixo esta descrito os materiais(hardware e software) utilizados neste projeto bem como o detalhamento do experimento.

\section{Materiais utilizados}
\subsection*{Openmote-B}
\subsection*{InfluxDB}

% \section{Dispositivos}
% \subsection{Transmissor}
% \subsubsection{Ciclo de envio de mensagens}
% \subsubsection{Conteudo das mensagens}
% \subsection{Receptor}
% \subsection{Gateway}
% \subsubsection{Leitor serial}
% \subsubsection{Persistência dos dados}