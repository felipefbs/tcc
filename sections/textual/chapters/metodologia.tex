\chapter{Experimento}
\label{experimento}

Neste trabalho, foi realizado um experimento em um ambiente de Smart Building, utilizando o padrão IEEE 802.15.4g SUN. A finalidade deste experimento é avaliar as performance da RSSF analisando valores de RSSI e PDR obtidos considerando uma rede formada por 11 dispositivos transmissores, denominados neste texto como Tx, e 3 dispositivos receptores, denominados neste texto como Rx.

O experimento foi realizado em um dos prédios do campus Campina Grande do Instituto Federal da Paraíba(IFPB). Sendo este constituído principalmente de salas de escritório e de alguns laboratórios, possuindo 4 andares separados por grossos pisos de concreto. Este cenário é particularmente desafiador para um enlace sem fio pois não é possível ter uma linha de visada entre o transmissor e o receptor além de apresentar muito ocorrência de propagação por multi-caminho e pode apresentar fácilmente áreas de sombreamentos de sinais.

Os dispositivos Tx foram espalhados pelos quatro andares do prédio, posicionados em canaletas e no interior de uma sala. Enquanto que os dispositivos Rx foram colocados no interior do laboratório GComPI, presente no primeiro piso do prédio. Os Tx enviavam mensagens nas três modulações SUN e cada Rx foi programado para receber mensagens de apenas uma modulação SUN. Sendo assim os 11 dispositivos Tx enviavam mensagens para cada um dos Rx que encaminhavam as mensagens e as informações de RSSI da mensagem para a porta serial que estavam conectados. Um script python rodando em um sistema operacional Linux captava as mensagens seriais, formatava e persistia estes dados em um banco de dados. Na seção \ref{funcionamento} será detalhada melhor o funcionamento de cada componente do experimento.

Abaixo esta descrito os materiais(hardware e software) utilizados neste projeto bem como o detalhamento do experimento.

\section{Materiais utilizados}
\subsection*{Openmote-B}
\subsection*{InfluxDB}

\section{Funcionamento da Rede}
\label{funcionamento}
