% resumo em português
\setlength{\absparsep}{18pt} % ajusta o espaçamento dos parágrafos do resumo
\begin{resumo}
  A Internet das Coisas possibilita uma forma de registrar nos mínimos detalhes tudo ao seu redor, como variações de temperatura em locais de uma cidade, bem como predizer se uma pessoa pode ou não ter problemas cardíacos a partir de sensores em seu corpo. A comunicação via rádio é uma solução para a implementação de sistemas IoT, porém este tipo de comunicação apresenta diversos desafios intrínsecos à sua utilização. Este trabalho se propõe a estender trabalhos anteriores, realizados em ambientes industriais e averiguar como a tecnologia de comunicação sem fio IEEE 802.15.4g Wi-SUN age em um ambiente predial, tendo em vista os diversos problemas que este tipo de ambiente causa em propagação de sinais de rádio. Para analisar, experimentalmente, o comportamento da tecnologia, foi implementada uma rede sem fio, na qual os dispositivos realizam cerca de 540 transmissões por hora, e os dados de suas transmissões foram armazenados e utilizados para analisar o desempenho da rede, a partir de parâmetros como taxa de entrega de pacotes e intensidade do sinal recebido. Como principal contribuição, este trabalho apresenta uma avaliação, em um ambiente real, do padrão IEEE 802.15.4g, mostrando como as particularidades das diferentes modulações são afetadas no ambiente predial. Como contribuição adicional, este trabalho cria e disponibiliza um conjunto de dados que podem ser utilizados por outros pesquisadores a fim de estudar e implementar técnicas de diversidade de modulação, possibilitadas graças aos diferentes esquemas de modulação disponíveis no 802.15.4g Wi-SUN.


  \textbf{Palavras-chaves}: IEEE 802.15.4g Wi-SUN. Redes Sem Fio. Internet das Coisas.
\end{resumo}

% resumo em inglês
\begin{resumo}[Abstract]
  \begin{otherlanguage*}{english}
    The Internet of Things provides a way to record everything around you in the smallest details, like the temperature variations in city locations, as well as predict whether or not a person may or may not have heart problems from sensors in their body. Radio communication is a solution for the implementation of IoT systems, however this type of communication presents several intrinsic challenges to its use. This work proposes to extend previous works, carried out in industrial environments, and to find out how the IEEE 802.15.4g Wi-SUN wireless communication technology acts in a building environment, considering the various problems that this type of environment causes in propagation radio signals. To analyze the behavior of the technology, a wireless network was implemented, in which the devices perform about 540 transmissions per hour, and the data of their transmissions were stored and used to analyze the performance of the network, using parameters such as  Packet Delivery Rate and Received Signal Strength Indication. As a main contribution, this work presents an assessment, in a real environment, of the IEEE 802.15.4g standard, showing how the particularities of the different modulations are affected in the building environment. As an additional contribution, this work creates and makes available a dataset that can be used by other researchers in order to study and implement modulation diversity techniques, made possible thanks to the different modulation schemes available in 802.15.4g Wi-SUN.


    \vspace{\onelineskip}

    \noindent
    \textbf{Key-words}: IEEE 802.15.4g Wi-SUN. Wireless Networks. Internet of Things.
  \end{otherlanguage*}
\end{resumo}

% resumo em francês 
% \begin{resumo}[Résumé]
%   \begin{otherlanguage*}{french}
%     Il s'agit d'un résumé en français.

%     \textbf{Mots-clés}: latex. abntex. publication de textes.
%   \end{otherlanguage*}
% \end{resumo}

% % resumo em espanhol
% \begin{resumo}[Resumen]
%   \begin{otherlanguage*}{spanish}
%     Este es el resumen en español.

%     \textbf{Palabras clave}: latex. abntex. publicación de textos.
%   \end{otherlanguage*}
% \end{resumo}
