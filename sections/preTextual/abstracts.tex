% resumo em português
\setlength{\absparsep}{18pt} % ajusta o espaçamento dos parágrafos do resumo
\begin{resumo}
  A integração que revolução da \emph{Internet of Things} está proporcionando está transformando a forma como humanos interagem com o mundo e possibilita uma forma de registrar nos mínimos detalhes tudo ao seu redor, desde as minimas variações de temperatura de inúmeros pontos em cidade, como predizer se uma pessoa pode ou não ter problemas do coração a partir de sensores no corpo do paciente. A comunicação via rádio é uma ótima solução para a implementação de sistemas IoT, porém este tipo de comunicação apresentam diversos desafios intrínsecos a sua utilização, e que precisam ser vencidos a fim de se ter uma ótima utilização das tecnologias de comunicação sem fio.
  Então este trabalho se propõe a estender trabalhos anteriores, realizados em ambientes industriais, e averiguar como a tecnologia de comunicação sem fio IEEE 802.15.4g Wi-SUN age em um ambiente prédial, tendo em vista os diversos problemas que este tipo de ambiente causa em propagação de sinais de rádio.
  Para analisar experimentalmente o comportamento da tecnologia, foi implementado uma rede sem fio, na qual os dispositivos realizam cerca de 540 transmissões por hora, e os dados de suas transmissões foram armazenados e utilizados para analisar da performance da rede, a partir de parâmetros como taxa de entrega de pacotes e intensidade do sinal recebido.
  Como principal contribuição, este trabalho apresenta uma implementação, em um ambiente real, do padrão IEEE 802.15.4g, mostrando como as particularidades das diferentes modulações são afetadas no ambiente prédial.
  Como contribuição adicional, este trabalho cria e disponibiliza um conjunto de dados que podem ser utilizados por outros pesquisadores a fim de estudar e implementar técnicas de diversidade de modulação, que é possibilitado graças aos diferentes esquemas de modulação presentes no 802.15.4g Wi-SUN.


  \textbf{Palavras-chaves}: IEEE 802.15.4g Wi-SUN. Redes Sem Fio. Internet das Coisas.
\end{resumo}

% resumo em inglês
\begin{resumo}[Abstract]
  \begin{otherlanguage*}{english}
    This is the english abstract.

    \vspace{\onelineskip}

    \noindent
    \textbf{Key-words}: latex. abntex. text editoration.
  \end{otherlanguage*}
\end{resumo}

% resumo em francês 
% \begin{resumo}[Résumé]
%   \begin{otherlanguage*}{french}
%     Il s'agit d'un résumé en français.

%     \textbf{Mots-clés}: latex. abntex. publication de textes.
%   \end{otherlanguage*}
% \end{resumo}

% % resumo em espanhol
% \begin{resumo}[Resumen]
%   \begin{otherlanguage*}{spanish}
%     Este es el resumen en español.

%     \textbf{Palabras clave}: latex. abntex. publicación de textos.
%   \end{otherlanguage*}
% \end{resumo}
