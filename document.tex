%Documento de Trabalho de Conclusão de Curso de Felipe Ferreira Bezerra da Silva do Curso Superior de Tecnologia em Telemática

\documentclass[
	% -- opções da classe memoir --
	12pt,				    % tamanho da fonte
	openright,			    % capítulos começam em pág ímpar (insere página vazia caso preciso)
	twoside,			    % para impressão em recto e verso. Oposto a oneside
	a4paper,			    % tamanho do papel. 
    % -- opções da classe abntex2 --
	%chapter=TITLE,		    % títulos de capítulos convertidos em letras maiúsculas
	%section=TITLE,		    % títulos de seções convertidos em letras maiúsculas
	%subsection=TITLE,  	% títulos de subseções convertidos em letras maiúsculas
	%subsubsection=TITLE,   % títulos de subsubseções convertidos em letras maiúsculas
    % -- opções do pacote babel --
	english,			    % idioma adicional para hifenização
	brazil				    % o último idioma é o principal do documento
    ]{abntex2}

% ---
% PACOTES
% ---
% Pacotes básicos 
\usepackage{babel}
\usepackage{lmodern}			                % Usa a fonte Latin Modern			
\usepackage[T1]{fontenc}		                % Selecao de codigos de fonte.
\usepackage[utf8]{inputenc}		                % Codificacao do documento (conversão automática dos acentos)
\usepackage{indentfirst}		                % Indenta o primeiro parágrafo de cada seção.
\usepackage{color}				                % Controle das cores
\usepackage{graphicx}			                % Inclusão de gráficos
\usepackage{microtype} 			                % para melhorias de justificação
% Pacotes de citações
\usepackage[brazilian,hyperpageref]{backref}	% Paginas com as citações na bibl
\usepackage[alf]{abntex2cite}
\usepackage{customizacao}	                % Citações padrão ABNT

% CONFIGURAÇÕES DE PACOTES
% Configurações do pacote backref
% Usado sem a opção hyperpageref de backref
\renewcommand{\backrefpagesname}{Citado na(s) página(s):~}
% Texto padrão antes do número das páginas
\renewcommand{\backref}{}
% Define os textos da citação
\renewcommand*{\backrefalt}[4]{
	\ifcase #1 
		Nenhuma citação no texto.
	\or
		Citado na página #2.
	\else
		Citado #1 vezes nas páginas #2.
	\fi}

%Minhas informações 
\titulo{Meu TCC}
\autor{Felipe Ferreira Bezerra da Silva}
\local{Campina Grande}
\data{CybeyPunk 2077}
\orientador{Orientador Fodão}
\newcommand{\instituto}{
    Instituto Federal da Paraiba 
    \par 
    Campus Campina Grande
    \par
    Curso Superior de Tecnologia em Telemática
}
\tipotrabalho{Trabalho de Conclusão de Curso}

\preambulo{Modelo canônico de trabalho monográfico acadêmico em conformidade com as normas ABNT apresentado à comunidade de usuários \LaTeX.}

\definecolor{blue}{RGB}{61, 126, 213}

% informações do PDF
\makeatletter
\hypersetup{
     	%pagebackref=true,
		pdftitle={\@title}, 
		pdfauthor={\@author},
    	pdfsubject={\imprimirpreambulo},
	    pdfcreator={LaTeX with abnTeX2},
		pdfkeywords={abnt}{latex}{abntex}{abntex2}{trabalho acadêmico}, 
		colorlinks=true,       		% false: boxed links; true: colored links
    	linkcolor=blue,          	% color of internal links
    	citecolor=blue,        		% color of links to bibliography
    	filecolor=magenta,      		% color of file links
		urlcolor=blue,
		bookmarksdepth=4
}
% Posiciona figuras e tabelas no topo da página quando adicionadas sozinhas
% em um página em branco. Ver https://github.com/abntex/abntex2/issues/170
\setlength{\@fptop}{5pt} % Set distance from top of page to first float
\makeatother

% O tamanho do parágrafo é dado por:
\setlength{\parindent}{1.3cm}

% Controle do espaçamento entre um parágrafo e outro:
\setlength{\parskip}{0.2cm}  % tente também \onelineskip

\begin{document}

	%Elementos Pré Textuais
	\pretextual
	\imprimircapa

	\imprimirfolhaderosto

	\begin{fichacatalografica}
	\sffamily
	\vspace*{\fill}					% Posição vertical
	\begin{center}					% Minipage Centralizado
        \fbox{\begin{minipage}[c][8cm]{13.5cm}		% Largura
            \small
            \imprimirautor
            %Sobrenome, Nome do autor
            
            \hspace{0.5cm} \imprimirtitulo  / \imprimirautor. --
            \imprimirlocal, \imprimirdata-
            
            \hspace{0.5cm} \thelastpage p. : il. (algumas color.) ; 30 cm.\\
            
            \hspace{0.5cm} \imprimirorientadorRotulo~\imprimirorientador\\
            
            \hspace{0.5cm}
            \parbox[t]{\textwidth}{\imprimirtipotrabalho~--~\imprimirinstituicao,
            \imprimirdata.}\\
            
            \hspace{0.5cm}
                1. Palavra-chave1.
                2. Palavra-chave2.
                2. Palavra-chave3.
                I. Orientador.
                II. Universidade xxx.
                III. Faculdade de xxx.
                IV. Título 			
        \end{minipage}}
	\end{center}
\end{fichacatalografica}
    \begin{folhadeaprovacao}

    \begin{center}
      {\ABNTEXchapterfont\large\imprimirautor}
  
      \vspace*{\fill}\vspace*{\fill}
      \begin{center}
        \ABNTEXchapterfont\bfseries\Large\imprimirtitulo
      \end{center}
      \vspace*{\fill}
      
      \hspace{.45\textwidth}
      \begin{minipage}{.5\textwidth}
          \imprimirpreambulo
      \end{minipage}%
      \vspace*{\fill}
     \end{center}
          
     Trabalho aprovado. \imprimirlocal, \imprimirdata:
  
     \assinatura{\textbf{\imprimirorientador} \\ Orientador} 
     \assinatura{\textbf{Professor} \\ Convidado 1}
     \assinatura{\textbf{Professor} \\ Convidado 2}
     %\assinatura{\textbf{Professor} \\ Convidado 3}
     %\assinatura{\textbf{Professor} \\ Convidado 4}
        
     \begin{center}
      \vspace*{0.5cm}
      {\large\imprimirlocal}
      \par
      {\large\imprimirdata}
      \vspace*{1cm}
    \end{center}
    
  \end{folhadeaprovacao}
    \begin{dedicatoria}
    \vspace*{\fill}
    \centering
    \noindent
    \textit{ Este trabalho é dedicado às crianças adultas que,\\
    quando pequenas, sonharam em se tornar cientistas.} \vspace*{\fill}
 \end{dedicatoria}
    \begin{agradecimentos}
    Os agradecimentos principais são direcionados à Gerald Weber, Miguel Frasson,
    Leslie H. Watter, Bruno Parente Lima, Flávio de Vasconcellos Corrêa, Otavio Real
    Salvador, Renato Machnievscz\footnote{Os nomes dos integrantes do primeiro
    projeto abn\TeX\ foram extraídos de
    \url{http://codigolivre.org.br/projects/abntex/}} e todos aqueles que
    contribuíram para que a produção de trabalhos acadêmicos conforme
    as normas ABNT com \LaTeX\ fosse possível.
    
    Agradecimentos especiais são direcionados ao Centro de Pesquisa em Arquitetura
    da Informação\footnote{\url{http://www.cpai.unb.br/}} da Universidade de
    Brasília (CPAI), ao grupo de usuários
    \emph{latex-br}\footnote{\url{http://groups.google.com/group/latex-br}} e aos
    novos voluntários do grupo
    \emph{\abnTeX}\footnote{\url{http://groups.google.com/group/abntex2} e
    \url{http://www.abntex.net.br/}}~que contribuíram e que ainda
    contribuirão para a evolução do \abnTeX.
    
    \end{agradecimentos}
    \begin{epigrafe}
    \vspace*{\fill}
	\begin{flushright}
		\textit{``Não vos amoldeis às estruturas deste mundo, \\
		mas transformai-vos pela renovação da mente, \\
		a fim de distinguir qual é a vontade de Deus: \\
		o que é bom, o que Lhe é agradável, o que é perfeito.\\
		(Bíblia Sagrada, Romanos 12, 2)}
	\end{flushright}
\end{epigrafe}
    \setlength{\absparsep}{18pt} % ajusta o espaçamento dos parágrafos do resumo
\begin{resumo}
 Segundo a \citeonline[3.1-3.2]{NBR6028:2003}, o resumo deve ressaltar o
 objetivo, o método, os resultados e as conclusões do documento. A ordem e a extensão
 destes itens dependem do tipo de resumo (informativo ou indicativo) e do
 tratamento que cada item recebe no documento original. O resumo deve ser
 precedido da referência do documento, com exceção do resumo inserido no
 próprio documento. (\ldots) As palavras-chave devem figurar logo abaixo do
 resumo, antecedidas da expressão Palavras-chave:, separadas entre si por
 ponto e finalizadas também por ponto.

 \textbf{Palavras-chave}: latex. abntex. editoração de texto.
\end{resumo}

% resumo em inglês
\begin{resumo}[Abstract]
    \begin{otherlanguage*}{english}
      This is the english abstract.
   
      \vspace{\onelineskip}
    
      \noindent 
      \textbf{Keywords}: latex. abntex. text editoration.
    \end{otherlanguage*}
   \end{resumo}

	% ---
% inserir lista de ilustrações
% ---
\pdfbookmark[0]{\listfigurename}{lof}
\listoffigures*
\cleardoublepage
% ---

% ---
% inserir lista de quadros
% ---
% \pdfbookmark[0]{\listofquadrosname}{loq}
% \listofquadros*
% \cleardoublepage
% ---

% ---
% inserir lista de tabelas
% ---
\pdfbookmark[0]{\listtablename}{lot}
\listoftables*
\cleardoublepage
% ---

% ---
% inserir lista de abreviaturas e siglas
% ---
\begin{siglas}
  \item[ABNT] Associação Brasileira de Normas Técnicas
  \item[abnTeX] ABsurdas Normas para TeX
\end{siglas}
% ---

% ---
% inserir lista de símbolos
% ---
\begin{simbolos}
  \item[$ \Gamma $] Letra grega Gama
  \item[$ \Lambda $] Lambda
  \item[$ \zeta $] Letra grega minúscula zeta
  \item[$ \in $] Pertence
\end{simbolos}
% ---

	
\end{document}